\section{Summary and discussion}
This thesis aimed to provide a new design for the playback and trace buffer that improves the old design on multiple fronts
\begin{itemize}
\item Usage of the complete $\DDRSIZE{}$ of storage available
\item The ability to reuse (parts of) already transmitted \PlaybackProgram{}s
\item The ability to read back the trace data in a different order than it was generated.
\item Deterministic timing for the execution of any \PlaybackProgram{}s split into blocks of atleast $\num{68} · \PhyWordSize{}$.
\end{itemize}
To achieve these goals the operation of the playback and trace buffer was redesigned from the ground. An \FPGA{} module that allows memory mapped access from the host and uses a scatter gather \DMA{} engine to construct the stream of playback instructions from multiple (potentially out of order) blocks was developed. Integration with the \BSSTwo{} software stack was performed. A new software layer called \ayo{} that is responsible for the low level interaction with the playback and trace buffer parts, such as reads and writes to the \DDR{} memory holding the playback and trace data aswell as the configuration of the \DMA{} engine was implemented. \ayo{} exposes an interface that allows usage of the new functionality, like the reuse of parts of the \PlaybackProgram{}s and partial readout of the trace data. Basic integration of \ayo{} with the \hxcomm{} layer was performed, that allows all current software to use the new playback and trace buffer design without any modifications, gaining the ability to use the much bigger $\DDRSIZE{}$ of storage available.

This new design for the playback buffer and software integration was tested extensively and compared in detail to the old design. For these tests and comparisons, the simulation environment was extended by a software accessible \AXI{} \DRAM{} and the \FPGA{} design was extended by a dummy data generator used to generate arbitrary amounts of trace data at maximum data rate.

It was verified that the software layer and memory mapped access to the \DDR{} memory is able to achieve a similar bandwidth between the host and the \FPGA{} as the old way of communication with the \FPGA{}.

The only disadvantage of the new playback and trace buffer is the reduced rate of experiments that can be performed in a \HWinTheLoop{} fashion.
For \PlaybackProgram{}s with a size of at least $\SI{1392}{\byte}$ and a trace data size of at least$\SI{1392}{\byte}$ the rate experiments that is possible with the new playback and trace buffer design was show to be less than two times lower than the rate for the old design. For any size of \PlaybackProgram{} and generated trace data, the rate of experiments was always less than $5$ times lower. The lower rate of experiments is caused by
additional \rtt{}s that are necessary between the \FPGA{} and the host and \autoref{sec:latency_reduction} outlines options for further extensions of this playback and trace buffer design to reduce the number of \rtt{}s required and therefore increase the rate of experiments again.
It was demonstrated that the complete size of the \DDR{} memory is possible to be used instead of only $\SI{64}{\mebi\byte}$ used by the old buffer design.

Finally, it was verified that for \PlaybackProgram{} constructed from blocks of at least $\num{68}·\PhyWordSize{}$ and for trace data organized into blocks of at least $\num{80}·\PhyWordSize{}$, the new playback and trace buffer always achieves the maximum possible bandwidth. This was verified to hold even for a variety of possible placements of these blocks in the \DDR{} memory space.
This constitutes a mayor improvement over the old playback and trace buffer design, which is not able to sustain this the maximum possible bandwidth for many \PlaybackProgram{} and or trace sizes and drastically improves the reliablity.

\section{Outlook}\label{sec:outlook}
The new memory mapped playback and trace buffer and the software integration presented in this thesis lays a foundation for a lot of future improvements of the \BSS{} stack.
\subsection{Latency reduction}\label{sec:latency_reduction}
The only disadvantage of the new playback and trace buffer is that for small \PlaybackProgram{}s the rate of experiments that can be performed in a \HWinTheLoop{} fashion is significantly lower than the rate of experiments that can be performed with the old design. This could be improved using one of several approaches
\subsection{Higher level software integration}
The \ayo{} software layer already exposes the new functionality of the new playback and trace buffer such as the ability to reuse block of already transmitted \PlaybackProgram{}s and partial readout of the generated trace data. However, this is functionality is not yet used by the upper layer of the \BSS{} software stack. As the rate with which experiments that can be performed is limited by the bandwidth between the host and the \FPGA{} for \PlaybackProgram{}s or trace data greater than $\SI{1392}{\byte}$, integration of this functionality with higher level software is expected to allow the rate of some experiments to be improved.
\begin{itemize}
  \item The \AXIDMA{} provides an interrupt signal that is asserted whenever the processing of a \descriptor{} is completed. These interrupts could be used to avoid the need for polling of the trace descriptor status.
  \item Introduction of a separate low latency path for the trace data, that bypasses the playback and trace buffer and instead is sent directly to the host, similar to the old trace buffer.
  \item Introduction of memory mapped access from the \FPGA{} to the host, to allow the \FPGA{} to write the host memory directly. This could for example be achieved by implementation of a module similar to \FAXI{} but operating is an opposite direction and providing an \AXI{} Subordinate interface to the \FPGA{}.
\end{itemize}
\subsection{Unified memory with PPU}
The \BSSTwo{} architecture includes microcontrollers on the \ASIC{} that allow for sophisticated on chip processing. They can for example be used for closed loop operation of the \ASIC{}. The \FPGA{} is connected to a \DDR{} memory, independent from the \DDR{} memory used for the playback and trace buffer, that provides working memory for these microcontrollers, that is used additionally to the on-chip \SRAM{}. In the futures it is envisioned to allow memory mapped access to this \DDR{} by the host using \FAXI{} and furthermore, the microcontrollers are envisioned to access the playback and trace memory, descriptor memory and \AXIDMA{} register space as well, to allow configuration directly from the microcontroller.
\subsection{Support of new host interfaces}
The old playback and trace buffer was tightly coupled to the stream interface of the \HostARQ{} protocol. By replacing the \FAXI{} module, the new playback and trace buffer design can be used with any host interface as long as one provides a bridge between the host interface and the \AXI{} bus. Using this one could for example use \PCIe{} instead of \UDP{} as the communication interface between the \FPGA{} and the host.
