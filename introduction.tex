\section{Introduction}
The field of neuromorphic computing draws inspiration from the structure and operations of the human brain to develop new approaches to computation. Neuromorphic computation devices can take many forms, such as digital simulation of neural systems on specialized digital computation devices\autocite{ref:spinnaker}. A different approach is the analog emulation of a neuron model, which promises efficiency and performance improvements over purely digital devices. This approach is taken by the BrainScaleS-2 (\BSSTwo{}) system\autocite{ref:bss2hw} developed by the Electronic Vision(s) Group of the Heidelberg University is the second generation of the \BSS{} neuromorphic computing platform. At its core it performs time-continuous analog emulation of the \AdEx{}\autocite{ref:adex} model for neurons. This is combined with sophisticated digital processing provided by microprocessors (PPU) extended with specialized SIMD units. The emulation is accelerated compared to the biological time by a factor of $\num{1000}$. The current silicon implementation of this architecture contains two of these microprocessors as well as $\num{512}$ neurons and $\num{512} · \num{256}$ synapses. This silicon realization will be called the \HICANNX{} hereafter. The \HICANNX{} is used in several modes of operation, like operation as a network attached accelerator\autocite{ref:network_accelerator} as well as usage as an edge computation platform\autocite{ref:mobile_system}.

Communication with the \HICANNX{} has real-time requirements, needs high bandwidth and requires precise timing, as the time continuous analog emulation performed by the \ASIC{} cannot be paused and resumed.

Current systems using the \HICANNX{} use an \FPGA{} as a bridge between a conventional host computer and the \ASIC{} to realize the real-time communication with the \ASIC{}. For this purpose the \FPGA{} includes a buffer. This buffer is used to store data prior to transmission to the \ASIC{} and store data received from the \ASIC{} until it is sent back to the host computer.

In this thesis a replacement design of this buffer is developed and implemented. It improves the reliability and increases the usable size while extending it with more functionality.
