\section{Introduction}
The field of neuromorphic computing draws inspiration from the structure and operations of the human brain to develop new approaches to computation. Neuromorphic computation devices can take many forms, such as digital simulation of neural system on specialized digital computation devices\autocite{ref:spinnaker}. One approach is the analog emulation of a neuron model, which promises efficiency and performance improvements of purely digital devices.

The \BSSTwo{} system developed by the Electronic Vision(s) Group of the Heidelberg University is the second generation of the \BSS{} neuromorphic computing platform. At its core combines analog emulation of the \AdEx{}\autocite{ref:adex} model for neurons time-continuous synapses with sophisticated digital processing like plasticity provided by microprocessors extended with specialized SIMD units. The emulation is accelerated compared to the biological time by a factor of $\num{1000}$. The current silicon implementation of this architecture is called \HICANNX{} and contains two of these microprocessors as well as $\num{512}$ neurons and $\num{512} · \num{256}$ synapses. The \HICANNX{} \ASIC{} can be used for a variety of applications, like operation as a network attached accelerator\autocite{ref:network_accelerator} as well as usage as a edge computation platform\autocite{ref:mobile_system}.

Communication with the \HICANNX{} \ASIC{} operates in a realtime fashion, needing a high bandwidth and precise timing, as the time continuous analog emulation performed by the \ASIC{} cannot be paused and continued arbitrarily.

Current systems using the \HICANNX{} \ASIC{} use a \FPGA{} as a bridge between a conventional host computer and the \ASIC{}, with the \FPGA{} facilitating the realtime communication. For this the \FPGA{} includes a buffer that is used to store data from host prior to transmission the \ASIC{} and store data received from the \ASIC{} prior to transmission back to the host computer.

This thesis investigates a replacement design of this buffer, that improves the reliability of the buffer, increases its usable size while extending it with more functionality.
